\documentclass{beamer}

\usepackage{beamerthemesplit}

\title{Specifying Programming Languages}
\author{Ryan Mulligan}
\date{\today}

\begin{document}

\frame{\titlepage}

\section[Outline]{}
\frame{
	\tableofcontents
}

\frame{
	\frametitle{Formal Definitions}
  
	A formal programming language definition allows you to reason mathematically---or formally---about your programming language. 
	In other words, formal definitions let you obtain a proof that goes from your program to a result.
}

\frame{
	\frametitle{Definitions}

	\begin{description}
	\item[Syntax] What the program looks like. Usually defined with a context free grammar.
	\item[Semantics] The meaning behind the syntax.
	\item[Specification] The syntax and semantics of a certain domain.
	\item[Program] An object valid under its intended syntax.
	\end{description}
}

\section{Uses}
\frame{
	\frametitle{Uses}

	\begin{itemize}
	\item Correctness
	\item Executable
	\item Deeper Understanding
	\end{itemize}
}

\subsection{Correctness}
\frame{
	\frametitle{Correctness}
	how a program behaves with respect to some specification
	\begin{itemize}
	\item partial
	\item total (Halt?)
	\end{itemize}
}

\subsection{Execution}
\frame{
	\frametitle{Execution}
	If your specification is written in the right way you can get an executable implementation for free. (K does this using Maude)

	Example: Maude and Piano Numbers
}

\begin{frame}[fragile]
	\frametitle{Maude's Grand Piano}
	Declarative and Executable
\begin{verbatim}
mod NAT is
  *** Syntax
  sort Nat .
  op 0 : -> Nat [ctor] .  
  op s : Nat -> Nat [ctor] .  
  op _+_ : Nat Nat -> Nat .
  *** Semantics
  vars N M : Nat .  
  eq 0 + s(N) = s(N) .  eq s(N) + M = s (N + M) .
endm
\end{verbatim}
\end{frame}

\subsection{Understanding}
\begin{frame}
	\frametitle{Understanding}
	If you specify a system you will be forced to have a deep understanding of it. Easier
	\begin{itemize}
	\item Implementation
	\item Debugging
	\end{itemize}
	and you can sleep at night, maybe...
\end{frame}

\section{Operational Semantics}
\frame{
  \huge{Structural Operational Semantics (SOS)}
}
\subsection{My Language}
\frame{
	\frametitle{A Simple Language}
	My programs are statements that store the result of arithmetic expressions into variables and it returns a variable.
	\begin{description}
	\item[$\sigma$] store
	\item[$a$] arithmetic expression
	\item[$i$] integer
	\item[$x$] a variable
	\end{description}
	Challenge: Write the BNF for this language. (If time.)
}

\subsection{Big Step}
\frame{
  \frametitle{Big Step SOS}
	A big step moves the program (P) from one state to another.

	$\frac{\langle a1, \sigma \rangle \Downarrow \langle i1 \rangle, \langle a2, \sigma \rangle \Downarrow \langle i2 \rangle }{ \langle a1 + a2, \sigma \rangle \Downarrow \langle i \rangle }$ where i is the sum of i1 and i2

	Challenge: Write rule for multiplication, division.
}

\subsection{Small Step}
\frame{
  \frametitle{Small Step SOS}
  A small step moves encapsulates a single computation.

  $\frac{\langle a1 , \sigma \rangle \longrightarrow \langle a1^\prime , \sigma \rangle }{ \langle a1 + a2 , \sigma \rangle \longrightarrow \langle a1^\prime + a2 , \sigma \rangle }$

	Challenge: What else do we need for Plus.

}

\section{Better Stuff}
\subsection{K}

\frame{
	\frametitle{K}
	$AExp, Statement, Program \leq K$\\
	$Int \leq KResult$\\
	$Config ::= Int | Program | state(State)$\\
	$K ::= K | K \curvearrowright K$
}

\frame{
	\frametitle{Operator Definitions}
	$\_ + \_ [strict\ extends +_Int ]$\\
  $ a1 + a2 \rightleftharpoons a1 \curvearrowright \Box + a2 $
}

\frame{
	\frametitle{Variables}
	$k(\frac{x}{\sigma[x]} \rangle state(\sigma)$\\
	$k(\frac{x:=i}{.} \rangle state(\frac{\sigma}{\sigma[x \leftarrow i]})$
}


\frame{
  \frametitle{Further}
	\begin{itemize}
	\item Come to SIGPLan
	\item Maude?
	\end{itemize}
}
\end{document}
    
